\documentclass[12pt]{article}%
%\geometry{verbose,letterpaper,tmargin=2.54cm,bmargin=2.54cm,lmargin=2.54cm,rmargin=2.54cm}
\usepackage{setspace}%
\usepackage{lineno}%
\usepackage{authblk}%
\usepackage[backend=bibtex,style= ele,natbib=true]{biblatex}%
\addbibresource{Stable_Isotopes_and_Fatty_acids.bib}%
\usepackage{mathtools}%
\usepackage{graphicx}%

\makeatletter%
\def\@maketitle{%
  \newpage
  \null
  \vskip 2em%
  \begin{center}%
  \let \footnote \thanks
    {\Large\bfseries \@title \par}%
    \vskip 1.5em%
    {\normalsize
      \lineskip .5em%
      \begin{tabular}[t]{c}%
        \@author
      \end{tabular}\par}%
    \vskip 1em%
    {\normalsize \@date}%
  \end{center}%
  \par
  \vskip 1.5em}
\makeatother


\begin{document}%
\begin{titlepage}%
\title{Bayesian estimation of predator diet composition from fatty
  acids and stable isotopes}%

\author{Philipp Neubauer*}
\thanks{Electronic address: \texttt{neubauer.phil@gmail.com}; Corresponding author}%
\affil{Dragonfly Science,\\PO Box 27535, Wellington 6141, New Zealand}%

\author{Olaf P. Jensen}%
\affil{Institute of Marine and Coastal Sciences\\Rutgers University, New Brunswick, NJ 08901, USA}%

\date{Dated: \today}%

\maketitle%

\end{titlepage}

\begin{spacing}{1.9} 
\begin{flushleft}

%\setcounter{page}{1}
\linenumbers
\begin{abstract}
Quantitative analysis of stable isotopes (SI) and, more recently,
fatty acid profiles (FAP) are useful and complementary tools for
estimating the relative contribution of different prey items in the
diet of a predator. The combination of these two approaches, however,
has thus far been limited and qualitative. We propose a mixing model
for FAP that follows the Bayesian machinery employed in
state-of-the-art mixing models for SI. This framework provides both
point estimates and probability distributions for individual and
population level diet proportions. Where fat content and conversion
coefficients are available, they can be used to improve diet
estimates. This model can be explicitly integrated with analogous
models for SI to increase resolution and clarify predator-prey
relationships. We apply our model to simulated data and an
experimental dataset that allows us to illustrate modeling strategies
and demonstrate model performance. Our methods are provided as an open
source software package for the statistical computing environment R.\\
\end{abstract}

\textbf{Keywords} Stable isotope analysis, quantitative fatty acid analysis,
QFASA, lipid profile, diet analysis, Bayesian mixing model, fatty acid
signature, dietary marker

\section{Introduction}
Quantitative estimates of an animal’s diet are a critical component of
predator-prey studies, ecosystem models, and ecosystem-based
management. Existing methods of estimating diet proportions all have
strengths and weaknesses \citep{bowen_methods_2012}. Traditional
stomach content and fecal matter analysis represent a brief snapshot
of diet at a particularly place and time and can be invasive,
time-consuming, and potentially biased by differential rates of
digestion of prey or ingestion of identifiable prey parts
\citep{bowen_methods_2012}. Chemical markers such as stable isotopes
(SI) and fatty acid profiles (FAP) solve some of these problems.  For
example, both approaches integrate diet composition over an extended
time period -  typically weeks to months, depending on tissue turnover
rates  \citep{tucker_convergence_2008}. These advantages have led to rapid
growth in the use of chemical markers in diet studies
 \citep{elsdon_unraveling_2010,williams_using_2010,kelly_fatty_2011,
bowen_methods_2012}. However, chemical dietary markers generally lack
the specificity of traditional stomach content analysis. In
particular, several prey species often have similar isotopic signatures. More
recent studies have sought greater dietary resolution through the use
of SI of other elements in addition to carbon and
nitrogen\citep{belicka_stable_2012}, compound specific SI ratios
\citep{budge_tracing_2008,jack_individual_2011}, or a combination of stomach
content analysis and SI or FAP \citep{pethybridge_seasonal_2012}. The
use of SI and FAP in combination also holds great promise; however the
few studies to date that have used both chemical markers have been
qualitative \citep{guest_trophic_2009} or based on positive correlation
of results from both methods \citep{tucker_convergence_2008}.

Analysis tools for SI data have become very sophisticated in recent
years, starting with the development of general Bayesian analysis
tools for estimating diet proportions, and leading to customized
(hierarchical) models for individual applications \citep{moore_incorporating_2008,hopkins_estimating_2012,parnell_bayesian_2012}. The latter models
can, for instance, estimate dietary differences of geographically
distinct populations \citep{semmens_quantifying_2009}, accommodate temporal
changes in diets or estimate the effect of covariates (e.g., age,
size, sex) on diet proportions \citep{parnell_bayesian_2012}.While these
models provide a considerable step towards ecologically relevant
models in diet studies, the underlying SI data is limited in the
resolution that it can provide. Since typically only 2-3 SI are
measured, the contrast that is achievable from such a low number of
variables is necessarily limited, especially when the number of
potential prey items increases \citep{phillips_source_2003,ward_quantitative_2011}. Optimally aggregating prey items into prey groups may
circumvent this problem \citep{ward_quantitative_2011}, but may also be less
satisfactory in complex food webs.

FAP data can, in theory, provide considerably more resolution compared
to SI data, due to large number of potential fatty acids that can be
measured. Nevertheless, studies employing FAP are either qualitative
in their estimates of prey proportions in predator diets, or use
Quantitative Fatty Acid Signature Analysis \citep{iverson_quantitative_2004} to
obtain quantitative estimates of diet proportions. The latter method
is the only one available thus far for use with FAP data, and, in
contrast to recent (Bayesian) SI mixing models, relies on a distance
metric rather than a model based formulation to estimate the most
likely diet proportions. This framework provided the first
quantitative approach to estimating diet proportions using fatty acids
and it has already seen widespread use, particularly in studies of
marine mammals \citep{bowen_methods_2012} and seabirds
\citep{williams_using_2010}.  Nevertheless, QFASA has a number of
limitations. Since it is not based on a probabilistic model, it is
difficult to estimate uncertainty associated with estimated diet
proportions \citep{williams_using_2010}. The absence of an explicit model
also makes it impossible to build ecological mechanisms (e.g.,
covariates of consumed diets) directly into the model. Furthermore,
uncertainty about conversion coefficients representing enrichment and
preferential uptake of fatty acids cannot be considered, yet small
changes in these coefficients can lead to differences in inferred diet
proportions \citep{wang_validating_2010}. Lastly, the QFASA model assumes
constant fat content of consumed items, an assumption that will rarely
be met.

Given the discrepancy in methods applied to SI and FAP data, it is
perhaps not surprising that their joint application has commonly
relied on qualitative comparisons. Because both markers integrate diet
composition over often comparable time-scales, however, an explicit
integration of these data types could provide substantial
benefits. While FAP data could mitigate the resolution problem in SI
data, SI data could provide increased resolution and clarify
predator-prey relationships, the knowledge of which is usually a
pre-requisite for FAP data. For example, for many non-modified fatty
acids, FAP alone cannot discriminate between the case of two species
which share a common diet and the situation in which one of these
species eats the other.  In either case, the two species may have
similar FAP.  The addition of a stable isotope with trophic
fractionation (e.g., $^{15}N$), however, can readily distinguish predation
from dietary overlap.

Here, we present a mixing model for FAP data based on a probabilistic
model whose parameters are estimated using Bayesian methods. We
demonstrate the suitability of this model for FAP analysis and
highlight the potential benefit of explicit integration with SI data
to increase the precision of diet estimates. Using both simulated and
published data, we show how this model can be extended to ask
ecologically relevant questions.
 

\section{Methods}
\subsection{A Bayesian mixing model for FAP}

Bayesian models for SI data are commonly based on the assumption that
SI ratios are normally distributed. This assumption cannot be made for
FAP data, since for most methods of analysis, the concentration of
individual fatty acids is normalized to the total lipid content of the
sample. Thus, the FAP are a collection of proportions (referred to as
a composition), which lie between 0 and 1, and are constrained to sum
to 1. A common solution to this problem, however, is to consider
transformations that make the data approximately normal
\citep{budge_studying_2006}. To construct our model, we followed
\citet{aitchison_convex_1999} and considered a log ratio transformation,
also called alr transformation, such that

\begin{align}
y_i = alr(\phi_i) = log \left( \frac{\phi_{i,1...p-1}}{\phi_{i,p}} \right)
\end{align}

where $y_i$ is the $p$-variate fatty acid composition of individual
$i$ of prey species $s$, with a total of $n$ potential prey species considered. We then assumed that the distribution of $y$ is
multivariate normal, with species specific mean $\mu_s$ and covariance matrix $\Sigma_S$,
or $y_i \sim N(\mu_s,\Sigma_s)$. A vaguely informative prior on
$\mu_s$ and $\Sigma_s$ allows for uncertainty in prey distributions
\citep{ward_including_2010} to propagate to estimates of diet proportions.

Each predator $j$ consumes a proportion $\pi_j$ of each prey source, and
analogous to stable isotope mixing models, predator FAP are then a
linear combination of prey FAPs, normalised to sum to one. Since predators may selectively assimilate or metabolize fatty acids
\citep{iverson_quantitative_2004,budge_studying_2006,rosen_effects_2012},
we specify prey-specific conversion coefficients $\kappa_i
=\kappa_{i,1}...\kappa_{i,P}$ \citep{rosen_effects_2012}. Furthermore,
the $n$ prey species likely have different fat content $\Phi$
that will affect the relative amount of fatty acids assimilated by the
predator. The FAP of predator $j$ is then a linear combination of the prey
FAP, modified by conversion coefficients for each fatty acid $p$ and fat
content for each prey $i$. The signature of predator $j$ is then:

\begin{align}
\label{eq:2}
\tau_{j} = C \left\{\sum_{s}^n  \left(\pi_{j,s} \Phi_{s} \right) \left(
    \kappa_{s} \otimes \phi_{s,j} \right) \right\}\\
t_j = alr(\tau_{j})

\end{align}

Here, $C$ is the closure operation which normalizes the FAP to sum to
one and $\otimes$ is the outer (element wise) product. $\phi_{i,j}$ is
the FAP of prey items of species $i$ consumed by predator
$j$. Similarly to \citet{parnell_bayesian_2012},
we thus assume that individual predators do not necessarily feed on
'average' (i.e,$\mu_s$) prey items, but rather consume prey items with
signatures drawn from the prey distribution $N(\mu_s,\Sigma_s)$. We
again assume that predator signatures are normally distributed after
transformation, such that $t_j ~ N(\mu_p,\Sigma_p)$. We assume that
$\Phi$ and $\kappa$ are log-normally and gamma
distributed, respectively, around known mean and variance values (estimated or
calculated from controlled diet experiments, see below). The
closure operation in \autoref{eq:2} (i.e., the sum-to-one constraint
on the FAP) leads to $\kappa$ being determined in terms of
relative uptake of fatty acids (i.e., up to a multiplicative
constant), and implicitly makes the multivariate distribution over all $\kappa$ a Dirichlet
distribution. The same logic applies to $\Phi$, and in both cases we
opted for formulations that can be readily parametrised from priors
studies or published values (e.g., sample means and variances).

The diet proportions $\pi$ predators are the main focus of investigation
in diet studies. It is equally possible to estimate individual diet
composition by simply  If data from individual predators are
available, but the focus remains on population level parameters, it is
generally advantageous to model individual  as draws from a population
level distribution of diet proportions . Recent approaches to stable
isotope mixing have focused on transformations of  to get around the
problems associated with the compositional nature of the diet
proportions. This approach is analogous to that taken in our model for
compositional FAPs. The diet proportions are transformed such that the
support of  is the real line rather than the interval [0;1]. It is
then straightforward to model diet proportions as function of
covariates, such as size, sex, or region (i.e., in a regression
formulation). While this approach is obviously appealing, it adds
considerably to the run-time of Markov Chain Monte Carlo Procedures
employed to estimate model parameters. We therefore use a vague
Dirichlet prior on the proportions when convenient (e.g., when we
estimate only population level parameters), and in our simulations
(e.g., Semmens et al. 2009). When estimating individual parameters or
for linear model formulations (see Application 1), a clr
transformation approach is used. When estimating population level
parameters, the two formulations give near identical results
(differences are within the range expected from  stochasticity in the
MCMC samples). 

\subsection{Joint diet estimation from FAP and SI}
Above, we mentioned three potential benefits of integrating FAP and SI
data: i) increased information to discriminate among sources, ii) the
potential of SI to resolve predator prey relationships due to trophic
enrichment of SI, and iii) the potential reduction in estimation error
due to more well-known fractionation coefficients for stable
isotopes. It thus appears worthwhile to integrate these types of data
in a single model to estimate diet proportions. Our model is
conceptually similar to recent models proposed for SI data, and
integration of FAP and SI data into a single model is straightforward
in the present setting. We again assume that the SI signatures of prey
items follow a normal distribution, such that , where the superscript
SI denotes that these are stable isotope signatures. Predator SI
signatures are again a linear combination of prey SI, this time
modified by additive fractionation coefficients and, potentially, by
prey C and N concentrations (and/or digestibility, see CITE) . The SI
signatures for predator j is then

The exact formulation of this integration depends on the assumptions that one is comfortable with in a given setting: identical dietary proportions may be appropriate if diets (and hence SI and FAP) are thought to be stable, or if both chemical tracers are thought to integrate over similar time-scales. If the time scales of these two elements are thought to be different, individual diet proportions for each tracer may be more appropriate, and may be drawn from an overall population distribution of diet proportions.
Our models were implemented in JAGS \citep{plummer_jags_2003}, called
from the statistical computing environment R \citep{R_core_2014}. Code
and data for all models, examples and tutorials are available on the
open source repository github.com/philipp-neubauer/fastinR. The models
include the above-mentioned formulations for individual diet
estimates, population level estimates or both as well as linear model
(regression and anova) formulations for diet proportions. 




\subsection{Simulation studies: accuracy and sensitivities}

We explored sensitivities of inferred diet proportions to the source
configuration and fatty acid subset selection in a series of
simulation experiments. We simulated 100 datasets and varied source
separation and the subset of fatty acids retained for the analysis of
simulated datasets. Each dataset was analyzed with subsets retaining
75\%-99\% of between source variability on the CAP axes, and errors in
inferred diet proportions (taken to be the posterior mean of inferred
diet proportions) were compared among subsets. We then determined the 
sensitivity of estimated diet proportions to source separation,
collinearity in FAP space and diet evenness (e.g., specialist versus 
generalist diets). Simulation setup and results are presented in
detail in Appendix 1 \& 2. To illustrate our method, we also include 
a simple simulated study, which can repeated in its entirety from 
Appendix 3 (this file can also be downloaded as a source file from our project repository).


\subsection{Application: Estimating predator diets in a controlled
  experiment}

In this application, we use data from an experimental study published
by \citet{stowasser_experimental_2006}, which investigated changes in squid FAP
and SI as a function on diet treatments. The treatments consisted of
exclusive fish and crustacean diets, as well as switched and mixed
diets, with the former switching diets from fish (henceforth SF) to
crustacean (SC) after 15 days of the 30 day experiment. In our
analysis, we analyzed samples from the switched diet treatments, and
used both SI and FAP to investigate whether we can infer diet
proportions in either treatments. Since we only had SI for the SC
treatment squid, we start by analyzing this treatment in isolation to
demonstrate that both SI and FAP can resolve diet proportions, and to
demonstrate the benefit of using the two tracers in a joint model. We
then analyze the SF and SC treatment squid together in a linear model
setup that investigates treatment differences explicitly, and
demonstrates how the model based approach to diet estimation can be
use to answer ecologically relevant questions about predator diets.

In order to apply our model, we first estimated conversion
coefficients of FAP and fractionation in SI, using squid from the 30
day diet treatments feeding exclusively crustacean and fish diets. The
model for estimation of SI fractionation followed the model in \citet{hussey_rescaling_2014}, and used their results as priors for fractionation
parameters for $\delta^{15}N$, and results from \citet{caut_variation_2009} to
construct priors for $\delta^{13}C$. Estimation of FA conversion coefficients used
eq 1 with proportions assumed known from feeding trials. Further
details on the estimation of conversion coefficients and fractionation
is given in Appendix 4.

\section{Results}

Simulated test cases and our application to squid diets suggest that our model can estimate diet proportions with high accuracy from both SI and FAP, with accuracy depending mainly on source separation and diet eveness (Appendix 1 \& 2). These examples further suggest that our selection procedure for FAP works well, allowed models at a fraction of computational cost with little expected loss in accuracy (Figure S1). At very uneven diet proportions, such as in the feeding trials analyzed in the squid example, we found the choice of point estimate for diet proportions inevitably introduced increased error at the margins of the 0-1 proportion interval. This is due to posterior distribution becoming more skewed toward the limits of this interval, and the mean and median, which are intuitive choices for point estimates in symmetrical posteriors, are often placed in relatively unlikely regions of parameter space.

\section{Discussion}

We presented here a first and very general framework that combines SI
and FAP in an extendable, contemporary Bayesian mixing model. While an
increasing number of studies combines these two tracer methods
\citep{tucker_convergence_2008, guest_evidence_2008,guest_trophic_2009,stowasser_experimental_2006,van_der_bank_dietary_2011,jaschinski_carbon_2008},
we believe that none have done so in an explicitly quantitative
way. Indeed, both approaches have their own limits, and, as mentioned
above, their combination may help to overcome each tracer’s
shortcomings. We thus suggest that our study and framework provide a
substantial step towards building application specific models that
explicitly integrate SI and FAP to achieve robust inference of diet
proportions and highlight discrepancies in the two methods that need
to be addressed through future research.

Recent developments in SI mixing models have led to increasingly
realistic models in terms of their error structure \citep{hopkins_estimating_2012} and
incorporation of relevant biology, such as time dependent diet
proportions and SI signatures \citep{parnell_bayesian_2012}. Given that our FAP and combined
FAP and SI models are very similar to these models in terms of their
underlying structure and assumptions, such developments are readily
achievable within this framework. Nevertheless, it should be noted
that they present the practitioner with requirements for substantial
amounts of data of various kinds (i.e., measurement error estimates,
collection of SI and FAP through time, respectively).

When working in high dimensional applications such as FAP, where the
number of measured variables can be large (>20 FAs is common), one has
to balance computation feasibility and model complexity. We opted here
for a fully Bayesian analysis that estimates prey and predator
distributions, as well as individual proportions. This does not come
without a cost: we found that there are limits to the dimensionality
that the model (as we formulated it) can deal with. Since the model
complexity depends at once on the number of prey items, predators and
fatty acids in the analysis, we have found it to be useful to use
predator means or relatively few predator signatures first to estimate
a single population distribution. Once one has determined that the
model can effectively estimate reasonable diet proportions, re-running
the model with a larger number of predators is warranted and, although
potentially time consuming, may provide additional insights. We also
found that using clr (or related) transformations to estimate diet
proportions on the real line \citep{parnell_bayesian_2012} was a considerable
computational burden, and we only used these formulations here when
the model structure would not work without such a formulation (i.e.,
hierarchical models of diet proportions, or a linear model/anova for
the diet proportions). Depending on modeling priorities, one may choose to only
model predator means, leading to lower computational burden and the
possibility to routinely use transformations for diet
proportions. Furthermore, more efficient implementations may be
possible with tailored and optimized MCMC approaches.

We presented an approach to variable selection for FAPs in order to
further reduce computational burdens of mixing models. Our suggested
method, based on CAP, provides a clear advantage over variable
selection by discrimination alone (e.g., by classification success in
a linear discriminant analysis). Discrimination is highest when the
within class variance is lowest relative to the between class
variance. A trivial selection would thus be to select a single FA that
is slightly different among prey species, but has minimal variance
within species. While classification accuracy would approach 100\%, n (the number of prey items) would be
greater than the number of selected variables, such that there is no
unique solution to the diet estimation problem \citep{phillips_source_2003}.
We found that an optimal subset of variables is usually one that
explains the bulk of among prey variance (represented by CAP axes),
but eliminates FAs that only contribute minimally to separation
among sources. In this case estimation errors may be lower than
for the full data set.
 

We refrain from making estimates of relative or absolute error of the
proportions themselves, since these depend on the number and
configuration of sources in multivariate space in addition to the
variance of conversion coefficients. For many applications, conversion
coefficients are not available and would be difficult to obtain, and
it is important that practitioners are aware of the increased risk of
making erroneous point estimates of diet proportions when setting
conversion coefficients to 1. While a similar problem exists with SI
fractionation coefficients, more is known about these coefficients,
making it easier to construct reasonable priors. Combining SI with FAP
may therefore have the additional benefit of reducing errors due to
misspecified conversion coefficients.


…. What else does this discussion need?
A general strategy for modeling FAP and SI data
Future research needs: discrepancies and overlap in SI and FAP data. – we assumed they integrate over comparable periods and thus represent the same diet proportions
CC for FAP – desperate need for LOTS of controlled studies. But, when taking into account uncertainty, they are not the deal breakers they appear to be from using just point estimates.


\printbibliography

\end{flushleft}
\end{spacing}

\end{document}